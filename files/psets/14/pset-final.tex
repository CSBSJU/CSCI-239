\documentclass[]{exam}

\usepackage[ruled,commentsnumbered,linesnumbered]{algorithm2e}
\usepackage{amsthm}
\usepackage{textcomp}

\pagestyle{empty}

\begin{document}
  \begin{center}
    \fbox{\fbox{\parbox{5.5in}{\centering Answer the questions in the spaces
    provided on the question sheets. If you run out of room for an answer,
    continue on a separate sheet of paper.}}}
  \end{center}

  \begin{questions}
    \question zyBooks 7.1.1 (a), (b), (f), and (g).
      \vspace{\stretch{1}}

    \question zyBooks 7.2.1
      \vspace{\stretch{1}}

    \question zyBooks 7.3.1 (a), (b), (e), (f), and (g).
      \vspace{\stretch{1}}

    \question zyBooks 7.3.1 (a), (b), (e), (f), and (g).
      \vspace{\stretch{1}}

    \question zyBooks 7.3.2 (a), (b), (c)
      \vspace{\stretch{1}}

    \question zyBooks 7.3.3
      \vspace{\stretch{1}}

    \newpage

    \uplevel{Consider the following algorithm for counting the triangles in a
      \emph{symmetric, non-reflexive} graph.\\[1\baselineskip]

      \begin{algorithm}[H]
        \DontPrintSemicolon
        \BlankLine
        \SetKwData{U}{U}
        \SetKwData{B}{B}
        \SetKwData{T}{T}
        \SetKwData{A}{A}
        \SetKwFunction{triu}{triu}
        \SetKwFunction{nnz}{nnz}
        \KwIn{An $n\times n$ \emph{adjacency matrix} called \A, of a
          \emph{symmetric, non-reflexive} graph $G$}
        \KwResult{The number of triangles in $G$}
        \BlankLine
        \tcc{\triu{\A} is $\A$ with only those edges (x,y) where x $\leq$ y.}
        $\U \gets \triu{\A}$ \;
        \;
        \tcc{$\U.\textnormal{\textquotesingle}$ is the transpose of \U and $\U *
          \U.\textnormal{\textquotesingle}$ is matrix multiplication.}
        $\B \gets \U * \U.\textnormal{\textquotesingle}$ \;
        \;
        \tcc{$\U \circ \B$ is the Hadamard product, which is similar to matrix
          addition, but multiplies rather than adds corresponding elements.}
        $\T \gets \U \circ \B$ \;
        \;
        \tcc{\nnz{\T} is the number of ones in \T.}
        \Return{\nnz{\T}}
        \caption{Triangle counting}
      \end{algorithm}

      \medskip
    }

    \question Analyze the algorithm \texttt{Triangle counting} and express the
      total number of additions, multiplications, and comparisons required for
      an $n \times n$ matrix as a function of $n$.
      \begin{solution}
        $f(n) = n^2 + 2n^3 + n^2 + 2n^2 = 2n^3+4n^2$

        $n^2$ comparisons for \texttt{triu}.\\
        $n^3$ multiplications and additions for matrix multiplication.\\
        $n^2$ multiplications for Hadamard product.\\
        $n^2$ comparisons and additions for counting ones.
      \end{solution}
      \vspace{\stretch{1}}

    \question Prove that this algorithm is $\Theta(n^3)$.
      \begin{solution}
        \begin{proof}
           Let $c = 6$ and $n_0 = 1$.

           For $n \geq 1$, $n^3 \geq n^2$, so

           $2n^3 + 4n^2 \leq 2n^3 + 4n^3 = 6n^3$

           Therefore for $n \geq 1$, $f(n) \leq 6n^3$.

           $f = O(n^3)$.
        \end{proof}

        \begin{proof}
           Let $c = 2$ and $n_0 = 1$.

           Since $n$ is positive, the $4n^2$ term in $f(n)$ is also positive.
           The positive terms can be dropped from the expression $2n^3 + 4n^2$
           and the resulting expression is smaller, so

           $2n^3 + 4n^2 \geq 2n^3$.

           Therefore for $n \geq 1$, $f(n) \geq 2n^3$.

           $f = \Omega(n^3)$.
        \end{proof}
      \end{solution}
      \vspace{\stretch{3}}
  \end{questions}
\end{document}
