\documentclass[]{exam}

\usepackage{amssymb}

\pagestyle{empty}


\def\bonuson{\renewcommand\partlabel{*(\thepartno)}}
\def\bonusoff{\renewcommand\partlabel{(\thepartno)}}

\begin{document}
  \begin{center}
    \fbox{\fbox{\parbox{5.5in}{\centering Answer the questions in the spaces
    provided on the question sheets. If you run out of room for an answer,
    continue on a separate sheet of paper.}}}
  \end{center}

  \begin{questions}
    \question Write a Haskell function to add the first \texttt{n} odd numbers
      of a list, using only the following functions: \texttt{sum},
      \texttt{filter}, \texttt{even}, \texttt{not} and \texttt{take}. The
      function signatures are given below.

      \texttt{sum :: Integral a => [a] -> a}\\
      \texttt{filter :: (a -> Bool) -> [a] -> [a]}\\
      \texttt{even :: Integral a => a -> Bool}\\
      \texttt{not :: Bool -> Bool}\\
      \texttt{take :: Integral -> [a] -> [a]}

      \vspace{\stretch{1}}

    \question Consider the following functions:

      $f: \mathbb{R} \rightarrow \mathbb{R}.~f(x)=x^2$\\
      $g: \mathbb{Z} \rightarrow \mathbb{R}.~g(x)=\frac{x}{2}$\\
      $h: \mathbb{R} \rightarrow \mathbb{Z}.~h(x)=\lceil x \rceil$

      \bigskip

      What will be the definition of the following function compositions

      \begin{parts}
        \part $(f \circ g)$
          \begin{solution}
            $(f \circ g): \mathbb{Z} \rightarrow \mathbb{R}.~(f \circ
            g)(x)=(\frac{x}{2})^2=\frac{x^2}{4}$
          \end{solution}
          \vspace{\stretch{1}}
        \part $(f \circ h)$
          \begin{solution}
            $(f \circ h): \mathbb{R} \rightarrow \mathbb{R}.~(f \circ
            g)(x)=(\lceil x \rceil)^2$
          \end{solution}
          \vspace{\stretch{1}}
        \part $(h \circ g \circ h \circ f)$
          \begin{solution}
            $(h \circ g \circ h \circ f): \mathbb{R} \rightarrow \mathbb{Z}.~(h
            \circ g \circ h \circ f)=\lceil \frac{\lceil x^2 \rceil}{2} \rceil$
          \end{solution}
          \vspace{\stretch{1}}

        \uplevel{Use your definitions to evaluate the following function
          compositions:}

        \part $(f \circ g)(1)$
          \begin{solution}
            $(f \circ g)(0) = \frac{1}{4}$
          \end{solution}
          \vspace{\stretch{1}}
        \part $(f \circ h)(3.5)$
          \begin{solution}
            $(f \circ h)(3.5) = 16$
          \end{solution}
          \vspace{\stretch{1}}
        \part $(h \circ g \circ h \circ f)(\sqrt 3)$
          \begin{solution}
            $(h \circ g \circ h \circ f)(\sqrt 3) = 2$
          \end{solution}
          \vspace{\stretch{1}}

      \newpage

      \uplevel{For each of the following functions, indicate whether the
        function has a well-defined inverse. If the inverse is well-defined,
        give the input/output relationship.}

        \part $f$
          \begin{solution}
            $f$ is not onto, thus is not bijective, thus is not invertible.
          \end{solution}
          \vspace{\stretch{1}}
        \part $g$
          \begin{solution}
            $g$ is not onto, thus is not bijective, thus is not invertible.
          \end{solution}
          \vspace{\stretch{1}}
        \part $(g \circ f)$
          \begin{solution}
            $f$ and $g$ cannot be composed, since the domain of $g$ is not a
            subset of the target of $f$, thus is not invertible.
          \end{solution}
          \vspace{\stretch{1}}
      \end{parts}
  \end{questions}
\end{document}
