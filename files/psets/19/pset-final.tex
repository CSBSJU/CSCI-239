\documentclass[]{exam}

\usepackage{amsmath}
\usepackage{amsthm}
\usepackage{graphicx}
\usepackage{subcaption}

\pagestyle{empty}

\begin{document}
  \begin{center}
    \fbox{\fbox{\parbox{5.5in}{\centering Answer the questions in the spaces
    provided on the question sheets. If you run out of room for an answer,
    continue on a separate sheet of paper.}}}
  \end{center}

  \section*{Counting by complement}

  \begin{questions}
    \question Give numerical answers for the questions below.
      \begin{parts}
        \part There are 5 kids on the math team. Two kids will be selected from
          the team to compete in the state competition. How many ways are there
          to select the 2 competitors? 
          \begin{solution}
            $\binom{5}{2}=\frac{5!}{2!(5-2)!}=\frac{5\cdot4\cdot3\cdot2\cdot1}{2\cdot1\cdot3\cdot2\cdot1}=\frac{5\cdot4}{2}=10$
          \end{solution}
          \vspace{\stretch{1}}

        \part The math team has 3 girls and 2 boys. How many ways are there to
          select the two competitors if they are both girls?
          \begin{solution}
            $\binom{3}{2}=\frac{3!}{2!(3-2)!}=\frac{3\cdot2\cdot1}{2\cdot1\cdot1}=\frac{6}{2}=3$
          \end{solution}
          \vspace{\stretch{1}}

        \part The math team has 3 girls and 2 boys. How many ways are there to
          select the two competitors so that at least one boy is chosen?
          \begin{solution}
            \textbf{Counting directly} Two choices for first boy. For each of
            those choices, there are four choices (1 boy, 3 girls) for the other
            competitor, which gives $2\cdot4=8$. However, counting this way
            counts the team of two boys twice, once for boy 1 and once for boy
            2, thus must be subtracted away, so $2\cdot4-1=7$.

            \textbf{Counting by complement} From Part (a), there are 10 total
            ways to make a two person team, three of which have no boys. Thus,
            the remaining $10-3=7$ teams must have at least one boy on them.
          \end{solution}
          \vspace{\stretch{1}}
      \end{parts}

    \newpage

    \uplevel{\section*{Inclusion-exclusion principle}}

    \question You are contracted by the FBI to unlock the phone of a suspected
      criminal. Your forensics team informs you that the material on the glass
      where the 8-key would appear during passcode entry has more oil-residue,
      left behind by fingers, than other places on the screen. This is
      understood to be a strong indication that the passcode contains at least
      one 8. How much does knowing that the passcode includes the digit 8 narrow
      the search space, considering that the passcode is 4 digits?

      \begin{parts}
        \part Use the inclusion-exclusion principle to count the number of 2
          digit passcode that include an 8.
          \begin{solution}
            There are 10 passcodes of the form \texttt{`8[0-9]'} and 10
            passcodes of the form \texttt{`[0-9]8'}. However these two sets are
            not disjoint. In fact their intersection has exactly one element,
            the passcode \texttt{`88'}. So to count the total number of 2 digit
            passcodes that include an 8, we have $10+10-1=19$.
          \end{solution}
          \vspace{\stretch{1}}

        \part Use the inclusion-exclusion principle to count the number of 3
          digit passcode that include an 8.
          \begin{solution}
            There are 100 passcodes of the form \texttt{`8[0-9][0-9]'}, 100
            passcodes of the form \texttt{`[0-9]8[0-9]'} and 100 passcodes of
            the form \texttt{`[0-9][0-9]8'}. However these three sets are not
            disjoint. Furthermore, no pair of these sets is disjoint. In fact,
            given any pair of these sets, their intersection has exactly 10
            elements and the intersection of all three sets has exactly one
            element. So to count the total number of 3 digit passcodes that
            include an 8, we have $100+100+100-10-10-10+1=271$.
          \end{solution}
          \vspace{\stretch{1}}

        \part Use the inclusion-exclusion principle to count the number of 4
          digit passcode that include an 8.
          \begin{solution}
            There are 1000 passcodes of the form \texttt{`8[0-9][0-9][0-9]'},
            likewise for an 8 in the second, third, or fourth position. As
            before, the sets are not disjoint, so we subtract away elements that
            are shared by any two sets, 100 per pair. Then we add back elements
            shared by any three sets, 10 per triple, than subtract away the
            single passcode that is in all four sets, namely \texttt{`8888'}.
            This gives us
            $1000+1000+1000+1000-\binom{4}{2}\cdot100+\binom{4}{3}\cdot10-1=4000-600+40-1=3439$.
          \end{solution}
          \vspace{\stretch{1}}
      \end{parts}

    \question Solve the previous three sub-problems using counting by complement
      rather than the inclusion-exclusion principle, to verify that you have the
      correct answers, i.e., the total number of passcodes should be the same.

      \begin{parts}
        \part~
          \begin{solution}
            The total number of 2 digit passcodes is $10^2$. The total number of
            2 digit passcodes that do not include an 8 is $9^2$. So the total
            number of 2 digit passcodes that include at least one 8 is
            $10^2-9^2=19$.
          \end{solution}
          \vspace{\stretch{1}}

        \part~
          \begin{solution}
            The total number of 3 digit passcodes is $10^3$. The total number of
            3 digit passcodes that do not include an 8 is $9^3$. So the total
            number of 3 digit passcodes that include at least one 8 is
            $10^3-9^3=271$.
          \end{solution}
          \vspace{\stretch{1}}

        \part~
          \begin{solution}
            The total number of 4 digit passcodes is $10^4$. The total number of
            4 digit passcodes that do not include an 8 is $9^4$. So the total
            number of 4 digit passcodes that include at least one 8 is
            $10^4-9^4=3439$.
          \end{solution}
          \vspace{\stretch{1}}
      \end{parts}
  \end{questions}
\end{document}
