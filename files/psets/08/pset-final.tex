\documentclass[]{exam}

\usepackage{amssymb}

\pagestyle{empty}


\def\bonuson{\renewcommand\partlabel{*(\thepartno)}}
\def\bonusoff{\renewcommand\partlabel{(\thepartno)}}

\begin{document}
  \begin{center}
    \fbox{\fbox{\parbox{5.5in}{\centering Answer the questions in the spaces
    provided on the question sheets. If you run out of room for an answer,
    continue on a separate sheet of paper.}}}
  \end{center}

  \noindent%
  You may find the following definitions helpful:\\[1\baselineskip]
  \noindent%
  $A \cap B = \{\ x\ :\ x \in A \land x \in B\ \}$\\
  $A \cup B = \{\ x\ :\ x \in A \lor x \in B\ \}$\\
  $A - B = \{\ x\ :\ x \in A \land x \notin B\ \}$\\
  $A \oplus B = \{\ x\ :\ x \in A \oplus x \in B\ \}$\\
  $A \times B = \{\ (a,\ b)\ :\ a \in A \land b \in B\ \}$\\
  $\overline{A} = \{\ x\ :\ x \notin A\ \}$

  \begin{questions}
    \begin{EnvUplevel}
      Consider the following sets:\\[1\baselineskip]

      $A = \{\ 1,\ 2,\ 3,\ 4,\ 5\ \}$\\
      $B = \{\ a,\ b,\ c,\ d,\ 4,\ 5\ \}$\\
      $C = \{\ a,\ b\ \}$
    \end{EnvUplevel}

    \bigskip

    \question Using roster notation, give formal descriptions of the following
      sets:

      \begin{parts}
        \part $A \cap B$
          \begin{solution}
            $\{\ 4,\ 5\ \}$
          \end{solution}
          \vspace{\stretch{1}}
        \part $A \cup B$
          \begin{solution}
            $\{\ 1,\ 2,\ 3,\ 4,\ 5,\ a,\ b,\ c,\ d\ \}$
          \end{solution}
          \vspace{\stretch{1}}
        \part $B - C$
          \begin{solution}
            $\{\ c,\ d,\ 4,\ 5\ \}$
          \end{solution}
          \vspace{\stretch{1}}
        \part $C - B$
          \begin{solution}
            $\varnothing$
          \end{solution}
          \vspace{\stretch{1}}
        \part $(A \cap B) \times C$
          \begin{solution}
            $\{\ (4,\ a),\ (4,\ b),\ (5,\ a),\ (5,\ b)\ \}$
          \end{solution}
          \vspace{\stretch{1}}
        \bonuson
        \part $\overline{A}$
          \begin{solution}
            $\mathbb{Z} - A$
          \end{solution}
          \vspace{\stretch{1}}
        \bonusoff
      \end{parts}

    \newpage

    \question For each of the following sets, draw the corresponding Venn
      diagram:

      \begin{parts}
        \part $A \cap B$
          \vspace{\stretch{1}}
        \part $A \cup B$
          \vspace{\stretch{1}}
        \part $B - C$
          \vspace{\stretch{1}}
        \part $C - B$
          \vspace{\stretch{1}}
        \part $\overline{A}$
          \vspace{\stretch{1}}
      \end{parts}

    \newpage

    \question For each of the following statements, let $|A| = n$ and $|B| = m$.
      If $A \subseteq B$, then what is the cardinality of each of the following
      sets:

      \begin{parts}
        \part $|A \cap B|$
          \begin{solution}
            $n$
          \end{solution}
          \vspace{\stretch{1}}
        \part $|A - B|$
          \begin{solution}
            0
          \end{solution}
          \vspace{\stretch{1}}
        \part $|B \oplus A|$
          \begin{solution}
            $m - n$
          \end{solution}
          \vspace{\stretch{1}}
        \part $|B \times A|$
          \begin{solution}
            $nm$
          \end{solution}
          \vspace{\stretch{1}}

        \uplevel{\emph{Hint:} for the following parts, you might find it useful
          to restate each set as an equivalent set, using applications of the
          \emph{set identities} before trying to determine cardinality}

          \bigskip\bigskip\bigskip

        \bonuson
        \part $|A \cup (A \cap B)|$
          \begin{solution}
            $A \cup (A \cap B) = A$, by \emph{absorption law}, so $|A \cup (A
            \cap B)| = |A| = n$.
          \end{solution}
          \vspace{\stretch{1}}
        \part $|(\overline{A} \cap B) \cup (A \cap B)|$
          \begin{solution}
            $(\overline{A} \cap B) \cup (A \cap B) = B \cap (A \cup
            \overline{A}) = B \cap U = B$, by \emph{distributive law},
            \emph{complement law}, \emph{identity law}, respectively, so
            $|(\overline{A} \cap B) \cup (A \cap B)| = |B| = m$.
          \end{solution}
          \vspace{\stretch{1}}
        \part $|A \cap (B \cap \overline{B})|$
          \begin{solution}
            $A \cap (B \cap \overline{B}) = A \cap \varnothing = \varnothing$,
            by \emph{complement law} and \emph{domination law}, respectively,
            so $|A \cap (B \cap \overline{B})| = |\varnothing| = 0$.
          \end{solution}
          \vspace{\stretch{1}}
        \bonusoff
      \end{parts}
  \end{questions}
\end{document}
