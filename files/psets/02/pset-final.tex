\documentclass[]{exam}

\pagestyle{empty}

\begin{document}
  \begin{center}
    \fbox{\fbox{\parbox{5.5in}{\centering Answer the questions in the spaces
    provided on the question sheets. If you run out of room for an answer,
    continue on a separate sheet of paper.}}}
  \end{center}

  \begin{questions}
    \question Using the laws of propositional logic, determine which of the
      following are equivalent to $(p \land q) \rightarrow r$ and which are
      equivalent to $(p \lor q) \rightarrow r$. Confirm your answer using
      truth-tables.
      % Have students split up the work, one member make the truth-tables for
      % even letters and use laws for odd letters, one member do the opposite,
      % and one member to coordinate and help each of the other two.

      \begin{parts}
        \part $p \rightarrow (q \rightarrow r)$
        \part $q \rightarrow (p \rightarrow r)$
        \part $(p \rightarrow r) \land (q \rightarrow r)$
        \part $(p \rightarrow r) \lor (q \rightarrow r)$
      \end{parts}

      \ifprintanswers\else
        \vspace{\stretch{1}}
      \fi

      \begin{solution}
        \begin{enumerate}
          \begin{minipage}{.45\textwidth}
            \item
              $p \rightarrow (q \rightarrow r)\\
               \lnot p \lor (q \rightarrow r)~(conditional~identity)\\
               \lnot p \lor \lnot q \lor r~(conditional~identity)\\
               \lnot (p \land q) \lor r~(deMorgan's)\\
               \equiv (p \land q) \rightarrow r~(conditional~identity)$
          \end{minipage}
          \begin{minipage}{.45\textwidth}
            \item
              $q \rightarrow (p \rightarrow r)\\
               \lnot q \lor (p \rightarrow r)~(conditional~identity)\\
               \lnot q \lor \lnot p \lor r~(conditional~identity)\\
               \lnot (p \land q) \lor r~(deMorgan's)\\
               \equiv (p \land q) \rightarrow r~(conditional~identity)$
          \end{minipage}

          \begin{minipage}{.45\textwidth}
            \item
              $(p \rightarrow r) \land (q \rightarrow r)\\
               (\lnot p \lor r) \land (\lnot q \lor r)~(cond.~identity)\\
               r \lor (\lnot p \land \lnot q)~(distributive)\\
               r \lor \lnot(p \lor q)~(deMorgan's)\\
               \equiv (p \lor q) \rightarrow r~(cond.~identity)$
          \end{minipage}
          \begin{minipage}{.45\textwidth}
            \item
              $(p \rightarrow r) \lor (q \rightarrow r)\\
               (\lnot p \lor r) \lor (\lnot q \lor r)~(cond.~identity)\\
               r \lor r \lor \lnot p \lor \lnot q~(commutative)\\
               r \lor \lnot p \lor \lnot q~(idempotent)\\
               r \lor \lnot(p \land q)~(deMorgan's)\\
               \equiv (p \land q) \rightarrow r~(cond.~identity)$
          \end{minipage}
        \end{enumerate}
      \end{solution}

    \question Convert the following English sentences into logical formulas. You
      may use expressions like $x=y$ or $x\neq y$ to indicate whether or not the
      variables $x$ and $y$ denote different people. The domain of discourse is
      all people. Let the predicate $H(x)$ mean that ``$x$ is happy,'' and let
      the predicate $L(x,y)$ mean that ``$x$ loves $y$.''

      \begin{parts}
        \part At least one person is happy.
        \part No one is happy.
        \part At least one person is unhappy.
        \part Exactly one person is happy.
        \part Not everyone loves someone else.
        \part Everyone loves someone else.
      \end{parts}

      \ifprintanswers\else
        \vspace{\stretch{1}}
      \fi

      \begin{solution}
        \begin{enumerate}
          \begin{minipage}{.45\textwidth}
            \item $\exists x H(x)$
            \item $\lnot \exists x H(x) \equiv \forall x (\lnot H(x))$
            \item $\exists x (\lnot H(x))$
            \item $\exists x (H(x) \wedge \forall y (H(y) \rightarrow x = y))$
          \end{minipage}
          \begin{minipage}{.45\textwidth}
            \item $\exists x \forall y(\lnot L(x,y))\\
              \lnot(\forall x \exists y L(x,y))\\
              \exists x \lnot (\exists y L(x,y))$
            \item $\lnot (\exists x \forall y(\lnot L(x,y))) \equiv \forall x
              \exists y L(x,y)\\
              \lnot(\lnot(\forall x \exists y L(x,y))) \equiv \forall x \exists
              y L(x,y)\\
              \lnot(\exists x \lnot (\exists y L(x,y))) \equiv \forall x \exists
              y L(x,y)$
          \end{minipage}
        \end{enumerate}
      \end{solution}

    \newpage

    \question Let the domain of discourse be all members of the class and let
      $L(x,y)$ be the predicate ``$x$ likes $y$.'' Translate the following
      into plain language:

      \begin{parts}
        \part $\forall x \exists y (L(x,y) \land x \neq y)$
        \part $\exists x \lnot \exists y (L(x,y) \lor L(y,x))$
        \part $\exists x \exists y \exists z \exists w (L(x,w) \land L(y,w)
          \land L(z,w) \land x \neq y \neq z)$
      \end{parts}

      \ifprintanswers\else
        \vspace{\stretch{1}}
      \fi

      \begin{solution}
        \begin{enumerate}
          \item Everyone in the class likes some other member of the class.
          \item There is a person who doesn't like anyone and who nobody likes.
          \item At least three different people like the same person.
        \end{enumerate}
      \end{solution}

    \question A certain cabal (\emph{cabal: a secret political clique or
      faction}) within the CS department is plotting to make the final exam
      \emph{ridiculously hard}. The only way to stop their evil plan is to
      determine exactly who is in the cabal. The department includes Donald,
      Grace, Linus, Alan, Ada and Edsger. The cabal is a subset of these six. A
      membership roster has been found and appears below, but it is deviously
      encrypted in logic notation. The predicate $cabal$ indicates who is in the
      cabal; that is, $cabal(x)$ is true if and only if $x$ is a member of the
      cabal. Use the following information to gather who is in the cabal. 

      \begin{enumerate}
        \item $\exists x \exists y \exists z (x \neq y \land x \neq z \land y
          \neq z \land cabal(x) \land cabal(y) \land cabal(z))$
        \item $\exists x (\lnot cabal(x))$
        \item $cabal(Edsger) \rightarrow \forall x (cabal(x))$
        \item $\lnot (cabal(Donald) \land cabal(Alan)) \land (cabal(Donald) \lor
          cabal(Alan))$
        \item $cabal(Alan) \rightarrow cabal(Donald)$
        \item $(cabal(Ada) \lor cabal(Linus)) \rightarrow \lnot(cabal(Grace))$
      \end{enumerate}

      \ifprintanswers\else
        \vspace{\stretch{1}}
      \fi

      \begin{solution}
        Donald, Ada, and Linus are in the cabal
      \end{solution}
  \end{questions}
\end{document}
