\documentclass[]{exam}

\pagestyle{empty}

\begin{document}
  \begin{center}
    \fbox{\fbox{\parbox{5.5in}{\centering Answer the questions in the spaces
    provided on the question sheets. If you run out of room for an answer,
    continue on a separate sheet of paper.}}}
  \end{center}

  \bigskip

  \begin{questions}
    \question Construct a truth table for the following compound propositions:
      \medskip
      \begin{parts}
        \begin{minipage}{.45\textwidth}
          \part $\lnot (p \lor q)$
            \begin{solution}
              \begin{tabular}{lll}
                $p$ & $q$ & $\lnot (p \lor q)$ \\
                0   & 0   & 1                  \\
                0   & 1   & 0                  \\
                1   & 0   & 0                  \\
                1   & 1   & 0
              \end{tabular}
            \end{solution}
        \end{minipage}
        \begin{minipage}{.45\textwidth}
          \part $\lnot (p \land \lnot (q \land s))$
            \begin{solution}
              \begin{tabular}{llll}
                $p$ & $q$ & $s$ & $\lnot (p \land \lnot (q \land s))$ \\
                0   & 0   & 0   & 1                                   \\
                0   & 0   & 1   & 1                                   \\
                0   & 1   & 0   & 1                                   \\
                0   & 1   & 1   & 1                                   \\
                1   & 0   & 0   & 0                                   \\
                1   & 0   & 1   & 0                                   \\
                1   & 1   & 0   & 0                                   \\
                1   & 1   & 1   & 1
              \end{tabular}
            \end{solution}
        \end{minipage}
      \end{parts}

      \vspace{\stretch{1}}

    \question
      \begin{parts}
        \part Using the propositions p=``I study'', q=``I will pass the
          course'', r=``The professor accepts bribes'',  translate the following
          into statements of propositional logic: 
          \begin{enumerate}
            \item If I do not study, then I will not pass the course unless the
              professor accepts bribes.
            \item If the professor accepts bribes, then I will pass the course
              regardless of whether or not I study.
            \item The professor does not accept bribes, but I study and will
              pass the course.
          \end{enumerate}

          \vspace{\stretch{1}}

          \begin{solution}
            \begin{enumerate}
              \item $(\lnot p \land q) \to r \equiv (\lnot p \land \lnot r) \rightarrow \lnot q$
              \item $r \rightarrow q$
              \item $\lnot r \land p \land q$
            \end{enumerate}
          \end{solution}
 
        \part Using the propositions p=``The night hunting is successful'',
          q=``The moon is full'', r=``The sky is cloudless'', translate the
          following into statements plain language:
          \begin{enumerate}
            \item $p \to (q \land r)$
            \item $\lnot r \leftrightarrow q$
            \item $(\lnot r \land p) \to q$
          \end{enumerate}

          \vspace{\stretch{1}}

          \begin{solution}
            \begin{enumerate}
              \item For successful night hunting it is necessary that the moon
                is full and the sky is cloudless.
              \item The sky being cloudy is both necessary and sufficient for
                the night hunting to be successful.
              \item  If the sky is cloudy, then the night hunting will not be
                successful unless the moon is full.
            \end{enumerate}
          \end{solution}
      \end{parts}

    \newpage

    \question Using a truth table, find which of the following compound
      propositions are always true (a tautology), regardless of the values of p
      and q: 
      \begin{enumerate} 
        \item $p \rightarrow (p \lor q)$
          \vspace{\stretch{1}}
        \item $p \rightarrow (p \rightarrow q)$
          \vspace{\stretch{1}}
        \item $\lnot (p \rightarrow (p \lor q))$
          \vspace{\stretch{1}}
      \end{enumerate}

      \begin{solution}
        \begin{enumerate} 
          \item Always true
          \item Not true for $p = 1, q = 0$
          \item Always false
        \end{enumerate}
      \end{solution}

    \question Find the inverse, converse and contrapositive for the following
      compound propositions, then evaluate each when $p$ is true and $q$ is
      false:
      \begin{enumerate} 
        \item $p \rightarrow \lnot q$
          \vspace{\stretch{1}}
          \begin{solution}
            inverse: $\lnot p \rightarrow q = T$\\
            converse: $\lnot q \rightarrow p = T$\\
            contrapositive: $q \rightarrow \lnot p = T$
          \end{solution}
        \item $\lnot p \rightarrow (p \lor q)$
          \vspace{\stretch{1}}
          \begin{solution}
            inverse: $p \rightarrow \lnot (p \lor q) = F$\\
            converse: $(p \lor q) \rightarrow \lnot p = F$\\
            contrapositive: $\lnot (p \lor q) \rightarrow p = T$
          \end{solution}
        \item $p \rightarrow (p \rightarrow q)$
          \vspace{\stretch{1}}
          \begin{solution}
            inverse: $\lnot p \rightarrow \lnot (p \rightarrow q) = T$\\
            converse: $(p \rightarrow q) \rightarrow p = T$\\
            contrapositive: $\lnot (p \rightarrow q) \rightarrow \lnot p = F$
          \end{solution}
      \end{enumerate}
  \end{questions}
\end{document}
