\documentclass[]{exam}

\usepackage{bm}
\usepackage{xcolor}

\pagestyle{empty}

\extrawidth{.5in}

\begin{document}
  \begin{center}
    \fbox{\fbox{\parbox{5.5in}{\centering Answer the questions in the spaces
    provided on the question sheets. If you run out of room for an answer,
    continue on a separate sheet of paper.}}}
  \end{center}

  \begin{questions}
    \question

      \begin{parts}
        \part Using a truth table, show that $p \oplus q \equiv \lnot (p \land
          q) \land (p \lor q)$.
          \label{part:truth}

          \begin{solution}
            \begin{tabular}{l|l|l|l}
              $p$ & $q$ & $p \oplus q$ & $\lnot (p \land q) \land (p \lor q)$ \\
              1   & 1   & 0            & 0 \\
              1   & 0   & 1            & 1 \\
              0   & 1   & 1            & 1 \\
              0   & 0   & 0            & 0
            \end{tabular}
          \end{solution}

          \vspace{\stretch{1}}

        \part Using the laws of propositional logic and the result from Part
          (\ref{part:truth}), show that $p \oplus q \equiv
          (p \land \lnot q) \lor (\lnot p \land q)$.

          \begin{solution}~\\
            \begin{array}[t]{lclr}
              p \oplus q & \equiv & \lnot (p \land q) \land (p \lor q) &
              \mbox{(Part (\ref{part:truth}))}\\
                  & \equiv & (\lnot p \lor \lnot q) \land (p \lor q) & \mbox{(De
                  Morgan's)}\\
                  & \equiv & (\mbox{T} \land (\lnot p \lor \lnot q)) \land
                  (\mbox{T} \land (p \lor q)) & \mbox{(identity)}\\
                  & \equiv & ((p \lor \lnot p) \land (\lnot p \lor \lnot q))
                  \land ((q \lor \lnot q) \land (p \lor q)) &
                  \mbox{(complement)}\\
                  & \equiv & (\lnot p \lor (p \land \lnot q)) \land (q \lor
                  (p \land \lnot q)) & \mbox{(distributive)}\\
                  & \equiv & (p \land \lnot q) \lor (\lnot p \land q) &
                  \mbox{(distributive)}
            \end{array}
          \end{solution}

          \vspace{\stretch{1}}
      \end{parts}

    \newpage

    \question A certain cabal (\emph{cabal: a secret political clique or
      faction}) within the CS department is plotting to make the final exam
      \emph{ridiculously hard}. The only way to stop their evil plan is to
      determine exactly who is in the cabal. The department includes Donald,
      Grace, Linus, Alan, Ada and Edsger. The cabal is a subset of these six. A
      membership roster has been found and appears below, but it is deviously
      encrypted in logic notation. The predicate $cabal$ indicates who is in the
      cabal; that is, $cabal(x)$ is true if and only if $x$ is a member of the
      cabal. Use the following information to gather who is in the cabal. 

      \begin{enumerate}
        \item $\exists x \exists y \exists z (x \neq y \land x \neq z \land y
          \neq z \land cabal(x) \land cabal(y) \land cabal(z))$
        \item $\exists x (\lnot cabal(x))$
        \item $cabal(Edsger) \rightarrow \forall x (cabal(x))$
        \item $\lnot (cabal(Donald) \land cabal(Alan)) \land (cabal(Donald) \lor
          cabal(Alan))$
        \item $cabal(Alan) \rightarrow cabal(Donald)$
        \item $(cabal(Ada) \lor cabal(Linus)) \rightarrow \lnot cabal(Grace)$
      \end{enumerate}

      %\def\arraystretch{1.4}
      %\begin{array}[t]{clr}
      %  \mbox{H.} & \lnot (cabal(Donald) \land cabal(Alan)) \land
      %    (cabal(Donald) \lor cabal(Alan)) & \mbox{(hypothesis 4)}\\
      %  \mbox{I.} & cabal(Donald) \oplus cabal(Alan) & \mbox{(Question 1,
      %    part (a) H)}\\
      %  \mbox{J.} & (cabal(Donald) \land \lnot cabal(Alan)) \lor (\lnot
      %    cabal(Donald) \land cabal(Alan)) & \mbox{(Question 1, part (b)
      %    I)}\\
      %  \mbox{K.} & (cabal(Donald) \land \lnot cabal(Alan)) &
      %    \mbox{(simplification J)}\\
      %  \mbox{L.} & \lnot (cabal(Donald) \land \lnot cabal(Alan))
      %    \rightarrow (\lnot cabal(Donald) \land cabal(Alan))&
      %    \mbox{(conditional identity J)}\\
      %  \mbox{M.} & \lnot cabal(Donald) \land cabal(Alan) & \mbox{(modus
      %    tollens K \& L)}\\
      %  \mbox{N.} & \bm{\lnot cabal(Donald)} & \mbox{(simplification M)}\\
      %  \mbox{O.} & \bm{cabal(Alan)} & \mbox{(simplification M)}\\
      %\end{array}

      \begin{EnvFullwidth}
          \begin{solution}~\\
            \def\arraystretch{1.4}
            \begin{array}[t]{clr}
              \mbox{A.} & \exists x (\lnot cabal(x)) & \mbox{(hypothesis 2)}\\
              \mbox{B.} & c \mbox{ is a particular element} \land \lnot cabal(c)
                & \mbox{(existential instantiation A)}\\
              \mbox{C.} & c \mbox{ is a particular element} &
                \mbox{(simplification B)}\\
              \mbox{D.} & \forall x (cabal(Edsger) \rightarrow cabal(x)) &
                \mbox{(hypothesis 3)}\\
              \mbox{E.} & \lnot cabal(c) & \mbox{(simplification B)}\\
              \mbox{F.} & cabal(Edsger) \rightarrow cabal(c) & \mbox{(universal
                instantiation C \& D)}\\
              \mbox{G.} & \bm{\lnot cabal(Edsger)} & \mbox{(modus tollens E \&
                F)}\\

              \mbox{H.} & \lnot (cabal(Donald) \land cabal(Alan)) \land
                (cabal(Donald) \lor cabal(Alan)) & \mbox{(hypothesis 4)}\\
              \mbox{I.} & cabal(Donald) \oplus cabal(Alan) & \mbox{(Question 1,
                part (a) H)}\\
              \mbox{J.} & (cabal(Donald) \land \lnot cabal(Alan)) \lor (\lnot
                cabal(Donald) \land cabal(Alan)) & \mbox{(Question 1, part (b)
                I)}\\
              \mbox{K.} & cabal(Alan) \rightarrow cabal(Donald) &
                \mbox{(hypothesis 5)}\\
              \mbox{L.} & \lnot cabal(Alan) \lor cabal(Donald) &
                \mbox{(conditional identity K)}\\
              \mbox{M.} & cabal(Donald) \land \lnot cabal(Alan) &
                \mbox{(disjunctive syllogism J \& L)}\\
              \mbox{N.} & \bm{cabal(Donald)} & \mbox{(simplification M)}\\
              \mbox{O.} & \bm{\lnot cabal(Alan)} & \mbox{(simplification M)}\\
              \color{red}{\mbox{P.}} & \color{red}{(cabal(Ada) \land
                cabal(Linus))~\lor~(cabal(Ada) \land cabal(Grace))~\lor~} & \mbox{\color{red}(hypothesis 1 \& G \& N
                \& O~}\\
                        & \color{red}{~~~~(cabal(Linus) \land cabal(Grace))} &
                        \color{red}{\mbox{and a lot of hard work)}}\\
              \mbox{Q.} & (cabal(Ada) \lor cabal(Linus)) \rightarrow \lnot
                cabal(Grace) & \mbox{(hypothesis 6)}\\
              \mbox{R.} & \lnot (cabal(Ada) \lor cabal(Linus)) \lor \lnot
                cabal(Grace) & \mbox{(conditional identity Q)}\\
              \mbox{S.} & (\lnot cabal(Ada) \land \lnot cabal(Linus)) \lor \lnot
                cabal(Grace) & \mbox{(De Morgan's R)}\\
              \mbox{T.} & (\lnot cabal(Ada) \lor \lnot cabal(Grace)) \land
                (\lnot cabal(Linus) \lor \lnot cabal(Grace)) &
                \mbox{(distributive S)}\\
              \mbox{U.} & \lnot cabal(Ada) \lor \lnot cabal(Grace) &
                \mbox{(simplification T)}\\
              \mbox{V.} & \lnot (cabal(Ada) \land cabal(Grace)) & \mbox{(De
                Morgan's U)}\\
              \mbox{W.} & (cabal(Ada) \land cabal(Linus)) \lor (cabal(Linus)
                \land cabal(Grace)) & \mbox{(disjunctive syllogism P \& V)}\\
              \mbox{X.} & \lnot cabal(Linus) \lor \lnot cabal(Grace) &
                \mbox{(simplification T)}\\
              \mbox{Y.} & \lnot (cabal(Linus) \land cabal(Grace)) & \mbox{(De
                Morgan's X)}\\
              \mbox{Z.} & cabal(Ada) \land cabal(Linus) & \mbox{(disjunctive
                syllogism W \& Y)}\\
              \mbox{AA.} & \bm{cabal(Ada)} & \mbox{(simplification Z)}\\
              \mbox{BB.} & \bm{cabal(Linus)} & \mbox{(simplification Z)}\\
              \mbox{CC.} & \bm{\lnot cabal(Grace)} & \mbox{(disjunctive
                syllogism U \& AA)}\\
            \end{array}
          \end{solution}
      \end{EnvFullwidth}

      \vspace{\stretch{1}}
  \end{questions}
\end{document}
