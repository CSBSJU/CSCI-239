\documentclass[]{exam}

\usepackage{amssymb}

\pagestyle{empty}


\def\bonuson{\renewcommand\partlabel{*(\thepartno)}}
\def\bonusoff{\renewcommand\partlabel{(\thepartno)}}

\begin{document}
  \ifprintanswers\else
    \begin{center}
      \fbox{\fbox{\parbox{5.5in}{\centering Answer the questions in the spaces
      provided on the question sheets. If you run out of room for an answer,
      continue on a separate sheet of paper.}}}
    \end{center}
  \fi

  \begin{questions}
    \question Using set-builder notation, give formal descriptions of the
      following sets:

      \begin{parts}
        \part The set of positive integers that are even.
          \ifprintanswers\else
            \bigskip\bigskip\bigskip
          \fi
          \begin{solution}
            $\{\ x \in \mathbb{Z}^{+}\ :\ x\%2=0\ \}$
          \end{solution}
        \part $\{\ -2,\ -1,\ 0,\ 1,\ 2\ \}$
          \ifprintanswers\else
            \bigskip\bigskip\bigskip
          \fi
          \begin{solution}
            $\{\ x \in \mathbb{Z}\ :\ |x| \leq 2\ \}$ or
              $\{\ x \in \mathbb{Z}\ :\ -2 \leq x \leq 2\ \}$
          \end{solution}
        \part $\{\ 3,\ 6,\ 9,\ 12,\ \dots\ \}$
          \ifprintanswers\else
            \bigskip\bigskip\bigskip
          \fi
          \begin{solution}
            $\{\ x \in \mathbb{Z}^{+}\ :\ x\%3=0\ \}$
          \end{solution}
        \part $\{\ -3,\ -1,\ 1,\ 3,\ 5,\ 7,\ 9\ \}$
          \ifprintanswers\else
            \bigskip\bigskip\bigskip
          \fi
          \begin{solution}
            $\{\ x \in \mathbb{Z}\ :\ x\%2=1 \land -3 \leq x \leq 9\ \}$
          \end{solution}
        \part $\{\ 0,\ 10,\ 20,\ 30,\ \dots,\ 1000\ \}$
          \ifprintanswers\else
            \bigskip\bigskip\bigskip
          \fi
          \begin{solution}
            $\{\ x \in \mathbb{Z}\ :\ x\%10=0 \land 0 \leq x \leq 1000\ \}$
          \end{solution}
        \bonuson
        \part The power set of $X$, denoted $P(X)$.
          \ifprintanswers\else
            \bigskip\bigskip\bigskip
          \fi
          \begin{solution}
            $\{\ A\ :\ A \subseteq X\ \}$
          \end{solution}
        \bonusoff
      \end{parts}

    \question State the cardinality of the following sets:

      \begin{parts}
        \part Question 1, part (a)
          \ifprintanswers\else
            \bigskip\bigskip\bigskip
          \fi
          \begin{solution}
            $\infty$
          \end{solution}
        \part Question 1, part (b)
          \ifprintanswers\else
            \bigskip\bigskip\bigskip
          \fi
          \begin{solution}
            5
          \end{solution}
        \part Question 1, part (d)
          \ifprintanswers\else
            \bigskip\bigskip\bigskip
          \fi
          \begin{solution}
            $\infty$
          \end{solution}
        \part Question 1, part (e)
          \ifprintanswers\else
            \bigskip\bigskip\bigskip
          \fi
          \begin{solution}
            101
          \end{solution}
        \part Question 1, part (f)
          \ifprintanswers\else
            \bigskip\bigskip\bigskip
          \fi
          \begin{solution}
            $2^{|X|}$
          \end{solution}
      \end{parts}

    \newpage

    \begin{EnvUplevel}
      Consider the following sets:

      $A = \{\ 2,\ 4,\ 6,\ 8\ \}$\\
      $B = \{\ x \in \mathbb{Z}\ :\ x \% 2 = 0 \land 0 < x < 10\ \}$\\
      $C = \{\ x \in \mathbb{Z}\ :\ x \% 2 = 0 \land 0 < x \leq 10\ \}$
    \end{EnvUplevel}

    \bigskip

    \question Indicate whether each statement about the sets $A$, $B$ and $C$ is
      true or false.

      \begin{parts}
        \part $A \subseteq B$
          \ifprintanswers\else
            \bigskip\bigskip\bigskip
          \fi
          \begin{solution}
            True. Every element of $A$ is also an element of $B$.
          \end{solution}
        \part $A \subset B$
          \ifprintanswers\else
            \bigskip\bigskip\bigskip
          \fi
          \begin{solution}
            False. $A = B$. There is no element of $B$ that is not also an
            element of $A$.
          \end{solution}
        \part $A \subseteq C$
          \ifprintanswers\else
            \bigskip\bigskip\bigskip
          \fi
          \begin{solution}
            True. Every element of $A$ is also an element of $C$.
          \end{solution}
        \part $A \subset C$
          \ifprintanswers\else
            \bigskip\bigskip\bigskip
          \fi
          \begin{solution}
            True. Every element of $A$ is also an element of $C$. Also $10 \in
            C$ and $10 \notin A$.
          \end{solution}
        \part $C \subseteq B$
          \ifprintanswers\else
            \bigskip\bigskip\bigskip
          \fi
          \begin{solution}
            False. $10 \in C$ and $10 \notin B$.
          \end{solution}
        \part $A = C$
          \ifprintanswers\else
            \bigskip\bigskip\bigskip
          \fi
          \begin{solution}
            False. $10 \in C$ and $10 \notin A$.
          \end{solution}
        \part $A = B$
          \ifprintanswers\else
            \bigskip\bigskip\bigskip
          \fi
          \begin{solution}
            True. $B = \{\ 2,\ 4,\ 6,\ 8\ \}$.
          \end{solution}
      \end{parts}

    \question Using roster notation, give formal descriptions of the
      following power sets:

      \begin{parts}
        \part $P(A)$
          \ifprintanswers\else
            \vspace{\stretch{1}}
          \fi
          \begin{solution}
            $P(A) = \{\ \varnothing,\ \{\ 2\ \},\ \{\ 4\ \},\ \{\ 6\ \}
                     ,\ \{\ 8\ \},\ \{\ 2,\ 4\ \},\ \{\ 2,\ 6\ \}
                     ,\ \{\ 2,\ 8\ \},\ \{\ 4,\ 6\ \},\ \{\ 4,\ 8\ \}
                     ,\\ \{\ 6,\ 8\ \},\ \{\ 2,\ 4,\ 6\ \},\ \{\ 2,\ 4,\ 8\ \}
                     ,\ \{\ 2,\ 6,\ 8\ \},\ \{\ 4,\ 6,\ 8\ \}
                     ,\ \{\ 2,\ 4,\ 6,\ 8\ \}\ \}$
          \end{solution}
        \part $P(B)$
          \ifprintanswers\else
            \vspace{\stretch{1}}
          \fi
          \begin{solution}
          \end{solution}
      \end{parts}
  \end{questions}
\end{document}
