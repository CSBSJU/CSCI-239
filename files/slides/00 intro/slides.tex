\documentclass[10pt,t,usenames,dvipsnames]{beamer}
\usetheme{metropolis}           % Use metropolis theme

\ifnotes
  \hypersetup{final}
  \usepackage{pgfpages}
  \setbeamertemplate{note page}[plain]
  \setbeameroption{show notes on second screen=right}
  % the following fixes bug which causes normal text to be white instead of
  % template default when notes are enabled
  % (see https://tex.stackexchange.com/questions/232168/normal-text-is-invisible-when-using-beamer-with-notes-and-xelatex)
  \makeatletter 
  \def\beamer@framenotesbegin{% at beginning of slide
    \usebeamercolor[fg]{normal text}
    \gdef\beamer@noteitems{}% 
    \gdef\beamer@notes{}% 
  }
  \makeatother
\fi

\usepackage{appendixnumberbeamer}
\usepackage[scale=3]{ccicons}   % creative commons icons
\usepackage{enumitem}
\usepackage{verbatim}

\title{CSCI 239 --- Discrete Computational Structures}
\date{}
\author{Jeremy Iverson}
\institute{College of Saint Benedict \& Saint John's University}
\begin{document}
  \maketitle

  \begin{frame}{logistics}
    \begin{itemize}[itemsep=10pt]
      \item instructor
        \begin{itemize}
          \item Jeremy Iverson (\href{mailto:jiverson002@csbsju.edu}{\nolinkurl{jiverson002@csbsju.edu}})
          \item PENGL 258, (320) 363-3083
          \item office hours: M 12:30--1:30pm, R 1:00--2:00pm
        \end{itemize}
      \item textbook
        \begin{itemize}
          \item \emph{Discrete Mathematics}, Irani et al., zyBook
        \end{itemize}
      \item website
        \begin{itemize}
          \item \url{https://csbsju.instructure.com/courses/12826}
        \end{itemize}
    \end{itemize}

    \note{
      \begin{itemize}[label=\textbf{\cdot}]
        \item I prefer to be called $(\{\varnothing, \mbox{Dr.}, \mbox{Prof.}\}\times\{\varnothing, \mbox{Jeremy}, \mbox{Iverson}\})-\{(\varnothing,\varnothing)\}$
        \item Encourage questions right away
        \item Emphasize the importance of the Canvas site for finding
          information about the class
        \item office hours
          \begin{itemize}
            \item Mention outlook calendar \& my home page
              \begin{itemize}
                \item For those unfamiliar with Outlook meetings, then they
                  should schedule another way and we will go over this in
                  meeting
              \end{itemize}
          \end{itemize}
        \item go through Canvas page organization quickly
      \end{itemize}
    }
  \end{frame}

  \begin{frame}{broad objectives}
    \begin{itemize}[itemsep=10pt]
      \item learn how to express ourselves precisely (read: unambiguously) using
        the language of logic
      \item learn the techniques necessary to establish logical certainty
      \item learn the fundamental discrete structures useful in computer science
        for reasoning about proposed solutions to computational problems
    \end{itemize}

    \note{
      \begin{itemize}
        \item logic
        \item proofs
        \item sets, functions, relations, growth of functions, induction,
          sequences, counting, discrete probability
      \end{itemize}
    }
  \end{frame}

  \begin{frame}[standout]{activity}
    \ifnotes
      \usebeamercolor[white]{normal text} % override bug fix in preamble
    \fi

    \begin{itemize}[itemsep=10pt]
      \item individually:
        \begin{itemize}[label=$\bullet$]
          \item complete the following sentences:
            \begin{enumerate}[label*=\arabic*.]
              \item This semester I hope to\dots
              \item My best advice for a first-year student to succeed
                academically is\dots
            \end{enumerate}
        \end{itemize}
      \item in groups of three:
        \begin{itemize}[label=$\bullet$]
          \item introduce yourselves to each other
          \item discuss your responses to the sentences above
          \item figure out what I prefer to be called
        \end{itemize}
    \end{itemize}

    \note{
      \begin{itemize}
        \item remind students to reflect on their hopes often during the
          semester
        \item remind students to take their own advice re succeeding in this and
          any other courses that they may be in this semester.
      \end{itemize}
    }
  \end{frame}

  \begin{frame}{mastery-based grading}
    \begin{alertblock}{simple idea}
      \begin{description}
        \item[traditional-grading] decompose assignments, quizzes, etc. into
          points --- mastery of learning objectives is demonstrated by
          accumulating a certain number of points.
        \pause
        \item[mastery-based grading] decompose assignments, quizzes, etc. into
          their representative learning objectives --- mastery of learning
          objectives is demonstrated by mastering learning objectives.
      \end{description}
    \end{alertblock}

    \begin{alertblock}{profound consequences}
      \begin{itemize}
        \item focused is shifted from \emph{how/when} you learn to \emph{what}
          you learn.
      \end{itemize}
    \end{alertblock}

    \note{
      \begin{itemize}
        \item how points are allocated and partial points awarded will influence
          what types of activities lead to good grades in class, not necessarily
          what learning objectives must be mastered to get desired grade.
          \begin{itemize}
            \item for example, if labs are weighted more heavily, then grades
              will generally be higher because labs are usually seen as learning
              activities and are graded on completion.
            \item however, if labs are weighted lower, because they are a
              learning activity and by the nature of being graded on completion
              do not necessarily represent mastery of learning objectives, then
              some other assessment type must be weighted higher. If exams are
              weighted higher, then students who struggle with exams may be
              disadvantaged and so on.
          \end{itemize}

        \item you are no longer bound/limited by how I allocated points. a
          particular grade is achieved by demonstrating mastery of a particular
          number of learning objectives rather than accumulating a particular
          number of points. this gives you much greater control over your grade
          and does a better job reflecting how much you learned from the course.

        \item standards-grading shifts focus from how mastery of learning
          objectives are assessed (read: how we allocate points) to which
          learning objectives are mastered. put simply, we stop looking at how /
          when you learn and focus on what you learn.
      \end{itemize}
    }
  \end{frame}

  \begin{frame}{our grading system}
    \begin{itemize}[itemsep=10pt]
      \item our learning objectives
      \item our grading specifications
      \item our schedule
    \end{itemize}

    \note{
      \begin{itemize}
        \item Canvas, as far as I know, is not capable of tracking grades in
          this way, so I WILL NOT be using the grading feature of Canvas. It
          will only be used to track completion of certain activities, i.e.,
          zyBook assignments, labs, etc.
      \end{itemize}
    }
  \end{frame}

  \begin{frame}[standout]
    \ifnotes
      \usebeamercolor[white]{normal text} % override bug fix in preamble
    \fi
    questions?
  \end{frame}

  \appendix

  \begin{frame}[standout]{activity}
    \ifnotes
      \usebeamercolor[white]{normal text} % override bug fix in preamble
    \fi

    \begin{itemize}
      \item download this slide deck and follow the instructions on the next
        slide
    \end{itemize}

    \note{
      \begin{itemize}
        \item if there is still time, talk about all of the material for this
          course being hosted on GitHub, and how they can access it
      \end{itemize}
    }
  \end{frame}

  \begin{frame}{next slide}
    \begin{itemize}
      \item find out what the reading is for Thursday
    \end{itemize}
  \end{frame}

  \begin{frame}[c]
    \begin{center}\ccbysa\end{center}

    except where otherwise noted, this worked is licensed under
    \href{http://creativecommons.org/licenses/by-sa/4.0/}{creative commons
    attribution-sharealike 4.0 international license}
  \end{frame}
\end{document}
