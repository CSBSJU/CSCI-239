\documentclass[]{exam}

\usepackage{amsthm}

\renewcommand{\questionshook}{%
  \setlength{\leftmargin}{0pt}%
  \setlength{\labelwidth}{-\labelsep}%
}
\renewcommand{\questionlabel}{$\Rightarrow$}
\renewcommand{\partlabel}{}
\renewcommand{\subpartlabel}{}

\newtheorem*{theorem}{Theorem}

\pagestyle{empty}

%Predicates and quantifiers
%
%    theorem
%    axioms
%        even number
%        odd number
%    valid argument
%        what doesn't this say about the conclusion
%    direct proof
%        p -> q

\begin{document}
  \begin{center}
    \fbox{\parbox{5.5in}{\centering Use the following spaces to record
      any information about key topics that you find useful.}}
  \end{center}

  \bigskip

  \begin{questions}
    \question Theorem:
      \medskip
      \begin{parts}
        \part Definition:
          \vspace{\stretch{1}}
        \part Some examples:
          \vspace{\stretch{1}}
      \end{parts}

    \question Axiom:
      \medskip
      \begin{parts}
        \part Definition:
          \vspace{\stretch{1}}
        \part Some examples:
          \vspace{\stretch{1}}
      \end{parts}

    \question \emph{Refresh} --- Argument:
      \medskip
      \begin{parts}
        \part Definition:
          \vspace{\stretch{1}}
        \part Stated as a proposition:
          \vspace{\stretch{1}}
          % p1 \land p2 \land \dots \land pn \rightarrow q
        \part Valid argument:
          \vspace{\stretch{1}}
          % An argument whose proposition is a tautology
      \end{parts}

    \newpage

    \question Direct proof:
      \medskip
      \begin{parts}
        \part Used to prove theorems of the form:
          \vspace{\stretch{1}}
          % p -> q
        \part Relationship between direct proofs and valid arguments:
          \vspace{\stretch{1}}
          % A valid argument means the condition proposition is a tautology, but
          % establishes nothing about the truthiness of the consequence.
        \part
          \begin{theorem}
            If $n$ is an even integer, then $n^2$ is even.
          \end{theorem}

          \begin{proof}
            \mbox{}\par
            \vspace{1in}
            \vspace{\stretch{1}}
          \end{proof}
      \end{parts}
  \end{questions}
\end{document}
