\documentclass[]{exam}

\usepackage{amssymb}
\usepackage{amsthm}

\renewcommand{\questionshook}{%
  \setlength{\leftmargin}{0pt}%
  \setlength{\labelwidth}{-\labelsep}%
}
\renewcommand{\questionlabel}{$\Rightarrow$}
\renewcommand{\partlabel}{}
\renewcommand{\subpartlabel}{}

\newtheorem*{theorem}{Theorem}
\newtheorem*{axiom}{Axiom}

\pagestyle{empty}

\begin{document}
  \begin{center}
    \fbox{\parbox{5.5in}{\centering Use the following spaces to record
      any information about key topics that you find useful.}}
  \end{center}

  \bigskip

  \begin{questions}
    \question Learning outcomes:
      \vspace{\stretch{2}}

    \question Function composition:
      \medskip
      \begin{parts}
        \part Key ideas:
          \vspace{\stretch{1}}
      \end{parts}

    \question Function inverse:
      \medskip
      \begin{parts}
        \part Expressed in set-builder notation:
          \vspace{\stretch{1}}
        \part Properties that a function must have in order for it to have an
          inverse:
          \vspace{\stretch{1}}
        \part How to find the inverse of a function:
          \vspace{\stretch{1}}
        \part Other key ideas:
          \vspace{\stretch{1}}
      \end{parts}

    \newpage

    \question Well-known functions:
      \medskip
      \begin{parts}
        \part floor:
          \vspace{\stretch{1}}
        \part ceiling:
          \vspace{\stretch{1}}
        \part exponential:
          \vspace{\stretch{1}}
        \part logarithm:
          \vspace{\stretch{1}}
      \end{parts}
  \end{questions}
\end{document}
