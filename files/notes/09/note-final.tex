\documentclass[]{exam}

\usepackage{amssymb}
\usepackage{amsthm}

\renewcommand{\questionshook}{%
  \setlength{\leftmargin}{0pt}%
  \setlength{\labelwidth}{-\labelsep}%
}
\renewcommand{\questionlabel}{$\Rightarrow$}
\renewcommand{\partlabel}{}
\renewcommand{\subpartlabel}{}

\newtheorem*{theorem}{Theorem}
\newtheorem*{axiom}{Axiom}

\pagestyle{empty}

\begin{document}
  \begin{center}
    \fbox{\parbox{5.5in}{\centering Use the following spaces to record
      any information about key topics that you find useful.}}
  \end{center}

  \bigskip

  \begin{questions}
    \question Learning outcomes:
      \vspace{\stretch{2}}

    \question Function:
      \medskip
      \begin{parts}
        \part Definition:
          \vspace{\stretch{1}}
        \part Some examples:
          \vspace{\stretch{1}}
        \part What is the relationship between a function and a set?
          \vspace{\stretch{1}}
          % a function is a set, i.e., (x, y) \in f, where f is a function,
          % furthermore, f \subseteq X \times Y, where x \in X and y \in Y.
      \end{parts}

    \question Function vocabulary:
      \medskip
      \begin{parts}
        \part Domain:
          \vspace{\stretch{1}}
        \part Target:
          \vspace{\stretch{1}}
        \part Range:
          \vspace{\stretch{1}}
      \end{parts}

    \newpage

    \question Function properties:
      \medskip
      \begin{parts}
        \part one-to-one (injective):
          \vspace{\stretch{1}}
        \part onto (surjective):
          \vspace{\stretch{1}}
        \part bijective:
          \vspace{\stretch{1}}
      \end{parts}
  \end{questions}
\end{document}
