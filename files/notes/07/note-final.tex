\documentclass[]{exam}

\usepackage{amssymb}
\usepackage{amsthm}

\renewcommand{\questionshook}{%
  \setlength{\leftmargin}{0pt}%
  \setlength{\labelwidth}{-\labelsep}%
}
\renewcommand{\questionlabel}{$\Rightarrow$}
\renewcommand{\partlabel}{}
\renewcommand{\subpartlabel}{}

\newtheorem*{theorem}{Theorem}
\newtheorem*{axiom}{Axiom}

\pagestyle{empty}

\begin{document}
  \begin{center}
    \fbox{\parbox{5.5in}{\centering Use the following spaces to record
      any information about key topics that you find useful.}}
  \end{center}

  \bigskip

  \begin{questions}
    \question Set:
      \medskip
      \begin{parts}
        \part Definition:
          \vspace{\stretch{1}}
        \part Some examples:
          \vspace{\stretch{1}}
        \part What distinguishes a set from other collections, like a list?
          \vspace{\stretch{1}}
          % no duplicates
      \end{parts}

    \question Set notation:
      \medskip
      \begin{parts}
        \part Roster notation:
          \vspace{\stretch{1}}
        \part Set builder notation:
          \vspace{\stretch{1}}
        \part Venn diagrams:
          \vspace{\stretch{1}}
      \end{parts}

    \newpage

    \question Some notable sets:
      \medskip
      \begin{parts}
        \part $\varnothing$:
          \vspace{\stretch{1}}
          % empty set
        \part $\mathbb{Z}$:
          \vspace{\stretch{1}}
          % all integers
        \part $\mathbb{N}$:
          \vspace{\stretch{1}}
          % natural numbers
        \part $\mathbb{Q}$:
          \vspace{\stretch{1}}
          % rational numbers
        \part $\mathbb{R}$:
          \vspace{\stretch{1}}
          % real numbers
        \part $\emph{U}$:
          \vspace{\stretch{1}}
          % universal set
      \end{parts}

    \question Cardinality:
      \medskip
      \begin{parts}
        \part Definition:
          \vspace{\stretch{1}}
        \part Notation:
          \vspace{\stretch{1}}
        \part Some examples of \emph{finite sets}:
          \vspace{\stretch{1}}
          % { 1, 2, 3, 4, 5 }, { n : n \in \N \land n < 3 }
        \part Some examples of \emph{infinite sets}:
          \vspace{\stretch{1}}
          % \Z, \N, \Q, \R
      \end{parts}

    \newpage

    \question Subsets:
      \medskip
      \begin{parts}
        \part Definition:
          \vspace{\stretch{1}}
        \part Some examples of subset relationships:
          \vspace{\stretch{1}}
        \part Relationship to set equality:
          \vspace{\stretch{1}}
      \end{parts}

    \question Proper subsets:
      \medskip
      \begin{parts}
        \part Definition:
          \vspace{\stretch{1}}
        \part Some examples of proper subset relationships:
          \vspace{\stretch{1}}
      \end{parts}

    \question Power sets:
      \medskip
      \begin{parts}
        \part Definition:
          \vspace{\stretch{1}}
        \part Notation:
          \vspace{\stretch{1}}
        \part Some examples:
          \vspace{\stretch{1}}
        \part Given the \emph{cardinality} of a set, what is the
          \emph{cardinality} of its power set?
          \vspace{\stretch{1}}
      \end{parts}
  \end{questions}
\end{document}
