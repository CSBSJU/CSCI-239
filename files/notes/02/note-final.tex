\documentclass[]{exam}

\renewcommand{\questionshook}{%
  \setlength{\leftmargin}{0pt}%
  \setlength{\labelwidth}{-\labelsep}%
}
\renewcommand{\questionlabel}{$\Rightarrow$}
\renewcommand{\partlabel}{}
\renewcommand{\subpartlabel}{}

\pagestyle{empty}

%Predicates and quantifiers
%
%    predicates
%    domains
%    quantifiers
%        universal ∀
%        existential ∃
%    quantified statements
%        free and bound variables
%        determining if a quantified statement is a proposition
%        understanding the meaning of a quantified statement
%        convert between logic and English
%    proving universal and existentially quantified statements
%        counterexamples
%    nested quantifiers

\begin{document}
  \begin{center}
    \fbox{\parbox{5.5in}{\centering Use the following spaces to record
      any information about key topics that you find useful.}}
  \end{center}

  \bigskip

  \begin{questions}
    \question Predicate:
      \medskip
      \begin{parts}
        \part Definition:
          \vspace{\stretch{1}}
        \part Some examples:
          \vspace{\stretch{1}}
        \part Key points about predicates:
          \vspace{\stretch{1}}
          % A predicate, P(x) is not a proposition, however, P(5) is a
          % proposition.
      \end{parts}

    \question Domain:
      \medskip
      \begin{parts}
        \part Definition:
          \vspace{\stretch{1}}
        \part Key points about domains:
          \vspace{\stretch{1}}
          % If a variables domain is not obvious, then it should be given as
          % part of the definition of the predicate.
      \end{parts}

    \question Universal quantifier:
      \medskip
      \begin{parts}
        \part Definition:
          \vspace{\stretch{1}}
          % equivalent proposition
        \part Some examples:
          \vspace{\stretch{1}}
        \part How to show a universally quantified statement is true?
          \vspace{\stretch{1}}
          % Show it is true for all values in the domain
        \part How to show a universally quantified statement is false?
          \vspace{\stretch{1}}
          % Find a single counterexample
      \end{parts}

    \newpage

    \question Existential quantifier:
      \medskip
      \begin{parts}
        \part Definition:
          \vspace{\stretch{1}}
          % equivalent proposition
        \part Some examples:
          \vspace{\stretch{1}}
        \part How to show a universally quantified statement is true?
          \vspace{\stretch{1}}
          % Find a single example that holds
        \part How to show a universally quantified statement is false?
          \vspace{\stretch{1}}
          % Show it is false for all values in the domain
      \end{parts}

    \question Precedence compared to logical connectives:
      \vspace{\stretch{1}}

    \question Bound and free variables:
      \medskip
      \begin{parts}
        \part How do they relate quantified statements to propositions?
          \vspace{\stretch{1}}
          % A proposition is a quantified statement with no free variables.
      \end{parts}
  \end{questions}
\end{document}
