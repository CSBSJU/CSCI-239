\documentclass[]{exam}

\usepackage{amsthm}

\renewcommand{\questionshook}{%
  \setlength{\leftmargin}{0pt}%
  \setlength{\labelwidth}{-\labelsep}%
}
\renewcommand{\questionlabel}{$\Rightarrow$}
\renewcommand{\partlabel}{}
\renewcommand{\subpartlabel}{}

\newtheorem*{theorem}{Theorem}
\newtheorem*{axiom}{Axiom}

\pagestyle{empty}

%Predicates and quantifiers
%
%    theorem
%    axioms
%        even number
%        odd number
%    valid argument
%        what doesn't this say about the conclusion
%    direct proof
%        p -> q

\begin{document}
  \begin{center}
    \fbox{\parbox{5.5in}{\centering Use the following spaces to record
      any information about key topics that you find useful.}}
  \end{center}

  \bigskip

  \begin{axiom}
    A rational number, is defined to be a number that can be expressed as
    the ratio of two integers in which the denominator is non-zero. 
  \end{axiom}

  \begin{questions}
    \question Proof by contrapositive:
      \medskip
      \begin{parts}
        \part Defined by which logical equivalence:
          \vspace{\stretch{1}}

        \part
          \begin{theorem}
            The square root of a positive real number is irrational if the
            number is irrational.
          \end{theorem}

          \bigskip

          \begin{proof}
            \mbox{}\par
            \vspace{\baselineskip}
            \textbf{Hypothesis}: \hrulefill

            \bigskip

            \textbf{Conclusion}: \hrulefill
            \mbox{}\par
            \vspace{\stretch{3}}
          \end{proof}
      \end{parts}

    \newpage

    \question Proof by contradiction:
      \medskip
      \begin{parts}
        \part Defined by the logical equivalence: $p \rightarrow q \equiv \lnot
          (p \rightarrow q) \rightarrow \mbox{F}$.

          \medskip

          Confirm this in the space below.
          \vspace{\stretch{1}}

        \part
          \begin{theorem}
            For all even integers $n$, $n^2$ is a multiple of 4.
          \end{theorem}

          \bigskip

          \begin{proof}
            \mbox{}\par
            \vspace{\baselineskip}
            \textbf{Hypothesis}: \hrulefill

            \bigskip

            \textbf{Conclusion}: \hrulefill
            \mbox{}\par
            \vspace{\stretch{3}}
          \end{proof}
      \end{parts}
  \end{questions}
\end{document}
